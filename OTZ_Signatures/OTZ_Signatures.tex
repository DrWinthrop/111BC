\documentclass{../signatures}
\labacronym{OTZ}
\labtitle{Optical Trapping}

\begin{document}

\maketitle

\names

\prelab

\begin{enumerate}

    \item How does an Optical Trap work?
    
    \item What is a Power Spectrum Density (PSD) Graph and what can it be used for?

    \item What is meant by sensitivity and stiffness of an Optical Trap?

    \item What are the safety requirements for working with this laser?
       \\[36pt]
\end{enumerate}

\prelabsignatures

\midlab

\begin{enumerate}

    \item By day 2 of this lab, you should have successfully created a slide with a dilute solution of 1 micron beads, turned on the laser, trapped a bead, and moved it vertically and horizontally. Start the Optical Trapping program and take real time data while the bead is trapped. Using the "Alt + Print Screen" command, copy the Power Spectrum Density graph and show it to a GSI.
\\[36pt]
\end{enumerate}

\midlabsignatures{2}

\pagebreak

\checkpointsection 

\begin{enumerate}

\item \checkp{Trapping a Single Bead}

\item \checkp{Sensitivity vs Laser Power Plot}

\item \checkp{Sensitivities From Alpha Values}

\item \checkp{Stiffness Values}

\item \checkp{Focus Questions}

\item \checkp{Two Sets of Questions}

\end{enumerate}

\end{document}