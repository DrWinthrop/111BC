\documentclass{../signatures}
\labacronym{MUO}
\labtitle{Muon Lifetime}

\begin{document}

\maketitle

\names

\prelab

\begin{enumerate}

    \item What is a muon?
    
    \item How and where are the muons in this experiment produced?

    \item In deriving the muon lifetime from the measured data, does any correction need be made for the time that the muon travels before it reaches the tank?

    \item The cosmic ray flux at sea level, integrated over all angles is approximately one particle per square centimeter per minute on any horizontal surface. The flux passing in both directions through a vertical surface is one-half as much. Use the zenith angle dependence of muon intensity to prove this result. Muons come only from the upper hemisphere (above the horizon). Based on this result, do you think the geometry of the detector matters.

    \item Given the geometry of the detector, 60 cm x 30 cm x 240 cm high, calculate the number of cosmic rays per minute which enter the detector.
    
    \item The fact that Figure 6 of Rossi \href{http://physics111.lib.berkeley.edu/Physics111/Reprints/MUO/02-Cosmic-Ray\_Phenomena.pdf}{Ref. 4} is relatively flat from zero to several hundred g/cm2, means that the number of muons which stop in a fairly shallow detector depends only on the mass of the detector. It does not depend on the shape of the detector, nor on the direction of incidence of the muons. For solid angle use 2$\pi$/3 steradians (=cos2θ integrated over the upper hemisphere), calculate the number of muons that will stop in the detector. Assume the density of the mineral oil is 0.8 g/cm3.

    \item Think how you will analyze the data. What program will you use? What steps need to be done? Discuss your choices with an instructor.
       \\[36pt]
\end{enumerate}

\prelabsignatures

\pagebreak

\midlab

\begin{enumerate}

    \item On day 3 of this lab, you should have successfully acquired an over-night muon spectrum with a calibrated time scale. Make a crude measurement of the lifetime. Show your spectrum to a GSI and ask for a signature.
\\[36pt]
\end{enumerate}

\midlabsignatures{3}

\checkpointsection 

\begin{enumerate}

\item \checkp{Oscilloscope (TDS 360)}

\item \checkp{Apparatus in the other room}

\item \checkp{Trigger Rate}

\item \checkp{Time Difference Data}

\end{enumerate}

\end{document}