\documentclass{../signatures}
\labacronym{HAL}
\labtitle{Hall Effect in Plasma}

\begin{document}

\maketitle

\names

\prelab

\begin{enumerate}

    \item What is the Hall Effect? Why do we examine the Hall Effect using plasma instead of a piece of metal?
    
    \item What does it mean to say that the plasma has a temperature? If the temperature is so high, why doesn’t the glass tube melt?

    \item What plasma parameters are you going to determine, and what measurements must you make besides the Hall voltage? To put another way, what are the relationships between what you measure and what you are going to calculate? For example, how do you get from a measurement of Hall voltage to a value of the electron density? Work out all these relationships now. Otherwise you might neglect to measure some relevant quantities.

    \item Approximately what potential do we apply across the tube to get a glow discharge?

    \item Why don’t we use a DVMM to measure the relevant voltages?
       \\[36pt]
\end{enumerate}

\prelabsignatures

\midlab

\begin{enumerate}

    \item On day 4 of this lab, you should have successfully produced a plot of EH vs B for at least one discharge-tube pressure value. Show it to a GSI and ask for a signature.
\\[36pt]
\end{enumerate}

\midlabsignatures{4}

\checkpointsection 

\begin{enumerate}

\item \checkp{Mean Electron Energy}

\item \checkp{Stable Plasma Flow}

\item \checkp{Valves and Probes}

\item \checkp{Shut off the System Completely}

\item \checkp{Hall Electric Field vs. Magnetic Field Plots}

\end{enumerate}

\end{document}