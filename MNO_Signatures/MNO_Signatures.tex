\documentclass{../signatures}
\labacronym{MNO}
\labtitle{Nonlinear Spectroscopy and Magneto-Optics}

\begin{document}

\maketitle

\names

\prelab

\begin{enumerate}

    \item Briefly describe the principle of operation of an external-cavity diode laser.
    
    \item What is a confocal Fabry-Perot spectrum analyzer? See the appendix on Spherical mirror Fabry-Perot. Define the following terms: free spectral range, finesse, and longitudinal and transverse modes. How do we achieve frequency tuning of the analyzer?

    \item What are the main mechanisms of spectral line broadening? Explain how it is possible to eliminate Doppler broadening using saturation spectroscopy.

    \item What are Faraday rotation and the Macaluso-Corbino effect? What is it that is nonlinear in the nonlinear Faraday effect?

    \item What are the safety requirements for working with this Laser?
       \\[36pt]
\end{enumerate}

\prelabsignatures

\midlab

By the third day of the lab:

\begin{enumerate}

    \item Demonstrate to a staff member your experimental plots of laser-induced fluorescence and Doppler-free absorption.
    
    \item Explain which features of your scans arise due to isotope shift and hyperfine structure.
\\[36pt]
\end{enumerate}

\midlabsignatures{3}

\checkpointsection 

\begin{enumerate}

\item \checkp{Optical Setup}

\item \checkp{Laser Beam Size and Shape}

\item \checkp{Frequency Range Scan}

\item \checkp{Vapor Cell}

\item \checkp{Earth's Magnetic Field}

\end{enumerate}

\end{document}