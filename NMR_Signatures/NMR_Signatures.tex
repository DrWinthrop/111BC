\documentclass{../signatures}
\labacronym{NMR}
\labtitle{Nuclear Magnetic Resonance}

\begin{document}

\maketitle

\names

\prelab

\begin{enumerate}

    \item What is nuclear magnetic resonance? What is resonating? What magnetic fields do we apply to our sample? What do these fields do? What is Larmor precession?
    
    \item Referring to Figure 6 of this lab manual, in what directions are the DC field, the modulating (60 Hz) field, and the RF field? What do these fields do? How do these fields relate to question 1? When you arrive in lab, examine how these fields are actually oriented in space. 

    \item What are T1 and T2?

    \item How does pulsed NMR differ from the continuous wave NMR experiment?

    \item What is "free nuclear induction"?

    \item What is a "spin echo"?
       \\[36pt]
\end{enumerate}

\prelabsignatures

\pagebreak

\midlab

\begin{enumerate}

    \item On day 3 of this lab, you should have successfully produced an H2O absorption resonance picture, with a calibrated frequency axis. What is the Larmor Frequency in MHz? Show them to a GSI and get a signature.
\\[36pt]
\end{enumerate}

\midlabsignatures{3}

\begin{enumerate}

\item On day 7 of this lab, you should have successfully observed the spin echo on the scope. Show it to an instructor and ask for a signature.
\\[36pt]
\end{enumerate}

\midlabsignatures{7}

\textbf{Other Questions to answer about this experiment as you go along}

\begin{enumerate}

    \item \textbf{Quantum Mechanics and E & M:} Classical absorption and dispersion curves for light going through matter (covered in any 110 text). In the NMR lab you will encounter similar absorption and dispersion curves. Why should optical absorption and dispersion be so similar to NMR absorption and dispersion? [Hint: think of the relevant Hamiltonians!]

    \item \textbf{E & M:} The experiment relies on the induced voltage generated in a "pickup" coil surrounding your sample (in more than one way). Study the sample NMR head apparatus and understand how a magnetization induced in your sample can be "picked up" by the coil (which is basically just an RLC circuit). Does the frequency of the RF field affect the pickup coil's response?
    \\[24pt]

\end{enumerate}

\pagebreak

\checkpointsection 

\begin{enumerate}

\item \checkp{Resonance Condition and Symmetry}

\item \checkp{Setup Pictures}

\item \checkp{CW Setup}

\item \checkp{Scanning Frequency}

\item \checkp{Mn Sample Traces}

\end{enumerate}

\end{document}